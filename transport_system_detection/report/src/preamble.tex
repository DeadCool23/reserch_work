\usepackage{cmap} % Улучшенный поиск русских слов в полученном pdf-файле
\usepackage[T2A]{fontenc} % Поддержка русских букв
\usepackage[utf8]{inputenc} % Кодировка utf8
\usepackage[english,russian]{babel} % Языки: русский, английский
\usepackage{enumitem} % Нумерованные и Маркированные списки
\usepackage{tabularx}
\usepackage{threeparttable} % Таблицы
\usepackage[14pt]{extsizes} % Шрифт 14 по ГОСТ 7.32-2001
\usepackage{caption} % Подписи к таблицам и рисункам
    \captionsetup{labelsep=endash} % Разделитель между номером и текстом
    \captionsetup[figure]{name={Рисунок}} % Подпись "Рисунок"
\usepackage{amsmath} % Мат. формулы

\usepackage{geometry} % Отступы
    \geometry{left=30mm}
    \geometry{right=10mm}
    \geometry{top=20mm}
    \geometry{bottom=20mm}
    \setlength{\parindent}{1.25cm} % Устанавливает отступ первой строки

\usepackage{titlesec} % Используется для форматирования заголовков
\titleformat{\section}
	{\normalsize\bfseries}
	{\thesection}
	{1em}{}
\titlespacing*{\chapter}{0pt}{-30pt}{8pt}
\titlespacing*{\section}{\parindent}{*4}{*4}
\titlespacing*{\subsection}{\parindent}{*4}{*4}

\usepackage{setspace} % Интервал
\onehalfspacing % Полуторный интервал

\frenchspacing % Промежутки между словами и предложениями отличаться не должны
\usepackage{indentfirst} % Красная строка

\usepackage{titlesec}
\titleformat{\chapter}{\LARGE\bfseries}{\thechapter}{20pt}{\LARGE\bfseries}
\titleformat{\section}{\Large\bfseries}{\thesection}{20pt}{\Large\bfseries}

\usepackage{multirow} % Объединение строк в таблице
\usepackage{listings} % Листинги
\usepackage{xcolor} % Цвета


\usepackage{pgfplots}
\usetikzlibrary{datavisualization}
\usetikzlibrary{datavisualization.formats.functions}

\usepackage{graphicx}

\usepackage[justification=centering]{caption} % Настройка подписей float объектов

\usepackage[unicode,pdftex]{hyperref} % Ссылки в pdf
\hypersetup{hidelinks}

\usepackage{csvsimple}
\newcommand{\code}[1]{\texttt{#1}}