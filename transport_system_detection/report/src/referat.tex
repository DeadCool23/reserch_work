\begin{center}
    \textbf{РЕФЕРАТ}
\end{center}
\setcounter{page}{3}

Расчетно-пояснительная записка к научно-исследовательской работе содержит 25 страниц, 10 иллюстраций, 2 таблицы, 16 источников, 1 приложение.

Научно-исследовательская работа представляет собой изучение предметной области выявления транспортной структуры на изображениях, описание основных методов, а также преимуществ и недостатков каждого из них. Рассмотрены детерминированный подход, экспертный подход и искусственные нейронные сети. Проведено сравнение различных модификаций нейронных сетей: семейства YOLO, семейства R-CNN, а также Single-Shot Detectors.

Ключевые слова: нейрон, перцептрон, сверточная нейронная сеть, капсульная нейронная сеть, R-CNN, SSD, YOLOv7.
