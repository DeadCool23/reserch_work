\begin{center}
    \textbf{ВВЕДЕНИЕ}
\end{center}
\addcontentsline{toc}{chapter}{ВВЕДЕНИЕ}

Распознавание объектов на изображениях является одной из наиболее распространенных задач при работе с визуальными данными.

Методы распознавания объектов на изображениях находят применение в самых разных областях. В системах управления транспортом они используются для повышения эффективности и безопасности дорожного движения. Видеоматериалы с камер наблюдения анализируются для определения элементов инфраструктуры, таких как перекрестки, туннели, мосты и дороги. В задачах картографирования, для оптимизации потоков транспорта и планирования маршрутов. В оцифровке рукописных текстов, упрощая обработку информации и ее дальнейшее использование.

\textbf{Цель научно-исследовательской работы:} анализ существующих методов решения задачи выявления транспортной структуры на изображениях.

Для достижения поставленной цели требуется выполнить следующие задачи:

\begin{itemize}
    \item[---] рассмотреть предметную область;
    \item[---] рассмотреть существующие методы решения задачи выявления транспортной структуры на изображениях;
    \item[---] провести сравнение рассмотренных методов.
\end{itemize}

\clearpage